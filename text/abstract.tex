\pagenumbering{roman}
\setcounter{page}{1}

\selecthungarian

%----------------------------------------------------------------------------
% Abstract in Hungarian
%----------------------------------------------------------------------------
\chapter*{Kivonat}\addcontentsline{toc}{chapter}{Kivonat}

Ahogy a beágyazott processzorok számítási kapacitása és a kommunikációs csatornák
sebessége növekszik, úgy válik egyre elterjedtebb és megengedhetőbb megoldássá
robotok alkalmazása. Különösen igaz ez olyan környezetekben amelyekben az emberi
munkaerő alkalmazása a környezet minősége, viszontagságai, vagy a munka
monotonitása miatt nem megengedhető. Ezen feladatok megoldása során gyakran van
szükségünk azonban a feladatot végző robot autonóm döntéseire, például navigáció
során. Az ilyen döntésre képes robotokat hívjuk autonóm robotoknak.

A diplomaterv céljaként egy autonóm robot megtervezését és megvalósítását tűztem
ki. A robot feladata egy előre ismert topológiájú labirintusban való haladás. Ez
a feladat egy könnyített szimulációja egy raktárépületben vagy
csatornarendszerben haladó robotnak. A robot megtervezése során az önálló
laboratóriumban elkezdett projektre építkezem.

A robot feladatai indokolják hogy nagyobb teljesítményű processzort használjak,
amely egy Linux operációs rendszer segítségével hajtja majd végre a feladatot.

Egy robot megvalósítása több kisebb modul együttes működését igényli. Egy ilyen
konstrukció kialakításához praktikus választás a ROS -- azaz Robot Operating
System -- keretrendszer felhasználása. A ROS egy robotikában open source library
csomag, amely a robotikában gyakran előforduló megoldások újrafelhasználását
teszi lehetővé.

A dolgozatban áttekintem a robot felépítését, fejlesztésének menetét. Bemutatom a
szükséges modulokat, és fejlesztésük menetét. Ismertetem a beágyazott Linux
rendszerek fordításának módját Yocto Project felhasználásával, valamint a ROS
keretrendszer integrálását a keresztfordítás folyamatába.

\vfill
\selectenglish


%----------------------------------------------------------------------------
% Abstract in English
%----------------------------------------------------------------------------
\chapter*{Abstract}\addcontentsline{toc}{chapter}{Abstract}

As the computational capabilities of embedded processors and the speed of
communication channels increase, the use of robots will become more common and
affordable. This is particularly true in environments where the quality,
harshness or monotonous nature of the work makes the use of human labour
unaffordable. However, when solving these tasks, we often need autonomous
decisions from the robot performing the task, for example during
navigation. Robots capable of making such decisions are called autonomous robots.

The goal of this thesis is to design and implement an autonomous robot. The
robot's task is to navigate through a maze with a known topology. This task is a
simplified simulation of a robot walking in a warehouse or a sewer system. In
designing the robot, I build on the project I started in a separate project lab.

The robot's tasks require me to use a more powerful processor that will execute
the task with a Linux operating system.

The implementation of a robot requires the combined operation of several smaller
modules. To design such a structure, it is a practical choice to use the ROS --
i.e. Robot Operating System -- framework. ROS is an open source library package
in robotics that allows reuse of solutions commonly found in robotics.

In this paper, I will review the design and development process of the robot. I
present the required modules and their development process. I will discuss how to
compile embedded Linux systems using Yocto Project and how to integrate the ROS
framework into the cross-compilation process.

\vfill
\selectthesislanguage

\newcounter{romanPage}
\setcounter{romanPage}{\value{page}}
\stepcounter{romanPage}
