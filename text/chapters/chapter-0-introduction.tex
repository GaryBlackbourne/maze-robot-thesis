%----------------------------------------------------------------------------
\chapter{\bevezetes}
%----------------------------------------------------------------------------

Napjaink fejlesztéseinek jelentős része automatizálási feladat, amely a
gyakorlatba ngyakran jelenti robotok alkalmazását. Számos fejlesztés törekszik
rá, hogy minél gyorsabb és hatékonyabb robotikai alkalmazásokat építhessünk.

Aktív munka folyik a hatékonyabb és nagyobb sebességű kód végrehajtásra képes
processzorok előállításán, vagy éppen a beágyazott környezet által igényelt
minél kisebb fogyasztású végrehajtó egységek elkészítésén. Más kutatások az
elosztott rendszerek hatékonyságának javításában játszanak nagy szerepet a
robosztusabb, nagyobb sebességű kommunikációs csatornák kiépítésével, mint
például az 5G hálózat.

\medskip

A robotika praktikus megoldásokat kínál számunkra olyan helyzetekben amikor az
emberi munkaerő nem alkalmazható, vagy kiváltásával olcsóbb vagy biztonságosabb
munkavégzés válik elérhetővé. Bizonyos feladatok megoldásához elkerülhetetlen
azonban, hogy a robot döntéseket legyen képes hozni a környezet adottságai
alapján. Ezek a döntések, és az általuk igényelt algoritmusok és processzorok
robotok alkalmazásakor gyakran jelentenek komplex feladatot.

Egy robot megtervezése és megépítése azonban nem csak a robot fő funkcióinak
megalkotásában jelent komoly feladatot. Egy komplex rendszer megtervezése és
megépítése szoftveres oldalról szemlélve is nehéz feladat, amely a mai napig
komoly kihívást jelent. Egy komplex rendszer számos komponens öszhangban való
működését igényli. Egy egyszerűbbnek tűnő funkció végrehajtásában is több
library és program megfelelő működése játszik szerepet. Egy teljes rendszerben az
egymástól függő csomagok és könyvtárak menedzselése komoly feladat, ami számos
emberi hibalehetőséget jelent, és ez, a rendszer karbantartásával, fejlesztésével
egyre csak növekedhet.

\medskip

A fenti problémára válaszul jelentek meg a buildrendszerek és verziókezelő
rendszerek. Az összetett build folyamatokban nagy segítséget jelent a folyamatok
automatizálása, valamint verziókezelő alkalmazás használata. Ezeknek az
eszközöknek a felhasználásával könnyebben állítható elő egy összetett
szoftvercsomag.

\section{A feladat}

Robotikai alkalmazásokban gyakran adódhat olyan szituáció, amikor olyan robotra
van szükség, amely képes egy előre megtervezett pályán haladni és eljutni annak
egy meghatározott pontjára. Ilyen helyzetekre lehet példa egy raktár logisztikai
rendszere, egy lakás takarítása, vagy akár egy csatornarendszer karbantartása.

\medskip

A diplomaterv feladat kiírásban szereplő labirintus egy ilyen jellegű problémát
hivatott modellezni. A feladat fő célja egy autonóm robot megalkotása. A robot
tervezése magában foglalja a robot koncepcionális terveit, általános
kialakításának meghatározását. A feladatban továbbá foglalkozom a robot hardveres
kialakításával, firmware fejlesztéssel mikrokontrolleres platformon, beágyazott
linux készítésével a Yocto project felhasználásával és a ROS2 keretrendszer
alapszintű megismerésével. Végül a felsorolt részfeladatok során a fejlesztés
és deployment automatizálásának kérdéskörével.

A feladat szerves részét képezi, hogy a projekthez mérten megismerem és bemutatom
a felhasznált keretrendszereket és eszközöket. A robotikai alkalmazások területén
a ROS, azaz Robot Operating System csomag egy elterjedt környezet, amely
segítségével a fejlesztés felgyorsítható, és moduláris, könnyen újrahasznosítható
kódot készíthetünk.


%% itt tartok:

A robot hardveres megépítését, valamint a firmware fejlesztését követően,
szükséges egy nagy teljesítményű operációsrendszer image előállítása is. Ez
beágyazott környezetben kimagasló többségben egy Linux alapú rendszer. Egy Linux
image előállítása manuálisan roppant nagy kihívás, ezért automatizált
rendszereket szokás használni.

el kell végezni a ROS2 applikáció futtatása között egy



\section{A Yocto project}

A beágyazott rendszerek területén nagyobb teljesítményű processzorok
felhasználására, valamint komplex rendszerek implementálására a legnépszerűbb
platform a Linux. Ennek a operációs rendszernek számos előnye van, amely miatt a
fent ismertetett megkötéseknek és igényeknek eleget tesz.

A robot fejlesztése során a magas szintü funkciók kialakítására alkalmas eszköz a
Robot Operating System csomag, amely egy DDS-en alapuló keretrendszer. Moduláris
könnyű felépítése és Linux rendszerbe integráltsága ideális eszközzé teszi, egy
fent ismertetett feladathoz.

A Diplomaterv során egy autonóm robot megtervezésének és felprogramozásának a
fázisait mutatom be. Elsőként a projekthez felhasznált alap projektet mutatom be,
amelyre a későbiek során építkezni fogok. Egyesével kitérek a központibb
felhasznált eszközök működésére úgy mint a felhasznált hardverek, és szenzorok, a
Linux rendszer és működése, a Yocto project és ROS keretrendszer, majd a
projektben történő felhasználásukat mutatom be.

\todo{ha elkészül összefoglalom hogy melyik fejezetben miről legyen szó}
\todo{inkább legyen több szó az előállítások módjáról, yocto, ros}
\todo{össze kéne foglalni a dolgozatban tárgyalt témákat, történési sorrendben}


