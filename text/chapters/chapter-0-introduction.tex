%----------------------------------------------------------------------------
\chapter{\bevezetes}
%----------------------------------------------------------------------------

Napjaink fejlesztéseinek jelentős része automatizálási feladat, amely a
gyakorlatba ngyakran jelenti robotok alkalmazását. Számos fejlesztés törekszik
rá, hogy minél gyorsabb és hatékonyabb robotikai alkalmazásokat építhessünk.

Aktív munka folyik a hatékonyabb és nagyobb sebességű kód végrehajtásra képes
processzorok előállításán, vagy éppen a beágyazott környezet által igényelt
minél kisebb fogyasztású végrehajtó egységek elkészítésén. Más kutatások az
elosztott rendszerek hatékonyságának javításában játszanak nagy szerepet a
robosztusabb, nagyobb sebességű kommunikációs csatornák kiépítésével, mint
például az 5G hálózat.

\medskip

A robotika praktikus megoldásokat kínál számunkra olyan helyzetekben amikor az
emberi munkaerő nem alkalmazható, vagy kiváltásával olcsóbb vagy biztonságosabb
munkavégzés válik elérhetővé. Bizonyos feladatok megoldásához elkerülhetetlen
azonban, hogy a robot döntéseket legyen képes hozni a környezet adottságai
alapján. Ezek a döntések, és az általuk igényelt algoritmusok és processzorok
robotok alkalmazásakor gyakran jelentenek komplex feladatot.

Egy robot megtervezése és megépítése azonban nem csak a robot fő funkcióinak
megalkotásában jelent komoly feladatot. Egy komplex rendszer megtervezése és
megépítése szoftveres oldalról szemlélve is nehéz feladat, amely a mai napig
komoly kihívást jelent. Egy komplex rendszer számos komponens öszhangban való
működését igényli. Egy egyszerűbbnek tűnő funkció végrehajtásában is több
library és program megfelelő működése játszik szerepet. Egy teljes rendszerben az
egymástól függő csomagok és könyvtárak menedzselése komoly feladat, ami számos
emberi hibalehetőséget jelent, és ez, a rendszer karbantartásával, fejlesztésével
egyre csak növekedhet.

\medskip

A fenti problémára válaszul jelentek meg a buildrendszerek és verziókezelő
rendszerek. Az összetett build folyamatokban nagy segítséget jelent a folyamatok
automatizálása, valamint verziókezelő alkalmazás használata. Ezeknek az
eszközöknek a felhasználásával könnyebben állítható elő egy összetett
szoftvercsomag.

\section{A feladat}

Robotikai alkalmazásokban gyakran adódhat olyan szituáció, amikor olyan robotra
van szükség, amely képes egy előre megtervezett pályán haladni és eljutni annak
egy meghatározott pontjára. Ilyen helyzetekre lehet példa egy raktár logisztikai
rendszere, egy lakás takarítása, vagy akár egy csatornarendszer karbantartása.

\medskip

A diplomaterv feladat kiírásban szereplő labirintus egy ilyen jellegű problémát
hivatott modellezni. A feladat fő célja egy autonóm robot megalkotása. A robot
tervezése magában foglalja a robot koncepcionális terveit, általános
kialakításának meghatározását. A feladatban foglalkozom továbbá a robot hardveres
kialakításával, a firmware fejlesztésével mikrokontrolleres platformon,
beágyazott linux készítésével a Yocto project felhasználásával és a ROS2
keretrendszer alapszintű megismerésével. Végül a felsorolt részfeladatok során a
fejlesztés és deployment automatizálásának kérdéskörével.

A feladat szerves részét képezi, hogy a projekthez mérten megismerem és bemutatom
a felhasznált keretrendszereket és eszközöket.

A robot magas szintű funkcionalitásának implementálása gyakran valamilyen magas
szintű nyelven történik. Ezek a funkciók illetve mőködési egységek gyakran
megegyeznek különböző robotokban, ezért megírásukat gyakran támogatja valamilyen
keretrendszer, ami a fejlesztést nagyban megkönnyítni. A feladat során egy ilyen
keretrendszert, a Robot Operating System-et, másnéven ROS-t használtam fel. A
robotikai alkalmazások területén a ROS, azaz Robot Operating System csomag egy
elterjedt környezet, amely segítségével a fejlesztés felgyorsítható, és
segítségével moduláris, könnyen újrahasznosítható kódot készíthetünk.

A robot hardveres megépítését, valamint a firmware fejlesztését követően,
szükséges egy nagy teljesítményű operációsrendszer, amelyen a magas szintű logika
megvalósulhat. Ez beágyazott környezetben kimagasló többségben egy Linux alapú
rendszer, amely mind nyílt forráskódja, mind hatékony és kevés erőforrást igénylő
működésének köszönhetően ideális választás. Egy Linux image előállítása
manuálisan roppant nagy kihívás és rengeteg hibalehetőséget hordoz, ezért a build
folyamatot automatizált rendszerekkel végezzük. A feladat megoldásában én a Yocto
projectet használtam.

\section{A Robot Operating System}

\todo{Ezt a részt bővíteni és részletezni.}
A ROS neve első hallásra megtévesztő, ez nem egy hivatalos értelemben vett
operációs rendszer, hanem sokkal inkább egy library és eszköz csomag, amely
robotikai alkalmazások fejlesztésére és futtatására lett létrehozva. A project
nagy előnye, hogy jól dokumentált, és nagy közösség alakult ki körülötte. 
Az open source megközelítésnek hála sok különböző hardverre lefordítható,
valamint fejlesztés során sok már a közösség által elkészített package, és
dokumentáció segíti a fejlesztőt.

\section{A Yocto project}

\todo{Ezt a részt bővíteni és részletezni.}
A Yocto project egy nyílt forráskódú projekt, amely linux disztribúciók
elkészítésében nyújt segítséget. A Yocto lényegében egy umbrella project, ami az
openembedded build system-et felhasználva és kiegészítve biztosít egy build
környezetet. A projekt sajátossága, hogy a build teljesen automatizált, a
fejlesztő a buildfolyamatot meghatározó konfigurációs fileok, és az egyes
build lépéseket meghatározó receptek szerkesztésével határozza meg a kívánt
disztribúció sajátosságait.

\section{A dolgozat tartalma}

A Diplomaterv dolgozat során egy autonóm robot megtervezésének fázisain haladok
végig. Elsőként a projekthez felhasznált alap hardver komponenseket mutatom be,
amelyre a későbiek során építkezni fogok. Kitérek ezek apróbb sajátosságaira, és
tervezés során tapasztalt javítási, fejlesztési lehetőségeikre. Ezután egyesével
részletezem a robot moduljait: a firmware-t, a linux disztribúciót, valamint a
ROS driver alkalmazást. A fejezetekben külön kitérek a modul kialakítására,
architektúrájára és a tervezés során figyelembe vett szempontokra; bemutatom a
felhasznált és-vagy létrehozott eszközöket és keretrendszereket; majd értékelem a
a produktum működésését, megállapítom a hibáit és kitérek fejlesztési
lehtőségeikre is.

\todo{ha elkészül összefoglalom hogy melyik fejezetben miről legyen szó}

