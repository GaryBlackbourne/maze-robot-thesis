%----------------------------------------------------------------------------
\chapter{ROS}
%----------------------------------------------------------------------------

Az előző fejezetekben bemutattam és végíghaladtam a robot egymásra épülő
szintjein. Ahogy a szintek egymásra épülnek, úgy növekszik az absztrakció, amely
nagy segítség abban, hogy az adott modul fejlesztése során a tényleges probléma
megoldására tudjunk koncentrálni.

A robot következő, és egyben utolsó modulja a Robot Operating System, a
legmagasabb absztrakciós szinten helyezkedik el. Ez a modul alkalmas a robot
fukcionalitásának implementálására, amennyiben a modulok, amelyekre alapul,
megfelelően működnek.

\section{A Robot Operating System}

A ROS, teljes nevén Robot Operating System, nevével ellentétben nem egy
hagyományos értelemben vett operációs rendszer, hanem egy könyvtár, program és
script csomag, amely robotikai alkalmazások fejlesztését és futtatását segíti.

A ROS célja, hogy a robot fejlesztéséhez minden szükséges komponenst biztosítson
a fejlesztő számára, hogy a fejlesztés során minél több időt lehessen a tényleges
problémára fordítani. Nagy előnye, hogy kialakításából adódóan nagyon jól
támogatja a kód újrafelhasználást, és a moduláris tervezést.

Ennek a tulajdonságának köszönhetően rengeteg package érhető el, amely különböző
funkcionalitásokat implementálnak a ROS rendszerben. A projekt teljesen open
source, ami szintén segítette az elterjedését. A fejlesztésben továbbá nagy
segítség a részletes dokumentáció.

\medskip

A ROS a Yocto Projecthez hasonlóan több kiadásban érhető el, amelyeket
disztribúciónak nevezünk. A disztribúciók időben egymást követik, és eltérő ideig
tartó támogatással rendelkeznek. A hosszú ideíg támogatott verziókat
LTS\footnote{LTS:~Long Term Support} disztribúciónak hívjuk. A projekt során a
stabilitás és elérhető dokumentáció miatt az aktuálisan legfrissebb LTS verziót
választottam, amelyet Humble Hawksbill-nek, vagy röviden csak
Humble-nek nevezik.

A Humble a projekt kezdetén nem rendelkezett stabil Yocto támogatással a
Kirkstone verzióhoz. Az ebből adódó hibák keresése nehézkessé tette a fejlesztés
kezdeti fázisát, de a projekt alatt a Kirkstone támogatás stabilizálódott, és a
rendszer a Yocto Project által buildelhetővé vált.

\subsection{Koncepciók}

A ROS platform a fejlesztést egy magasabb absztrakciós szinten közelíti meg, mint
az keretrendszerek többsége. Egy általános keretrendszer rendszerint egy adott
nyelven történő szoftverfejlesztést támogat extra funkcionalitással, amelyeket
könyvtárak formájában biztosít a fejlesztők számára.

Ezzel szemben a ROS egy node-oknak nevezett egységegből és ezek közötti
kommunikációs csatornákból álló rendszer megvalósításában nyújt segítséget. A
nodeok nem kapcsolódnak szigorúan egy futtatható programhoz, egy rendszert sok
kisebb futtatható bináris is alkothat. A fejlesztés nyelve sincsen szigorúan
megkötve, az adott komponensek fejlesztésére akár különböző nyelveket is
használhatunk, a közösség ráadásul számos nyelv támogatásán is dolgozik.

A projekt két nyelvet támogat hivatalosan, a C++-t és a Python-t. A fejlesztés
rendszerint ezeken a nyelveken zajlik.

\subsubsection{Node}

A ROS alapvető építőegysége a node. Egy node a robot valamely funkcióját
valósíthatja meg, és interfészek segítségével kapcsolódik a rendszer többi
node-jához. A node kód szinten egy objektumként jelenik meg, amely a ROS Node
osztályából származik le. Egy node futtatásához azonban valamilyen futtatható
binárisra, vagy python scriptre is szükség van, amely implementálja az osztályt,
és futásban tartja azt.

A nodeok rendszerint egyetlen jól körülhatárolható funkcionalitást valósítanak
meg, mint például egy szenzor olvasása, vagy egy motorvezérlési feladat. Egy
node-ot nem csak egyszer lehet példányosítnai, hanem több alkalommal is, egy jól
megtervezett node a funkcionalitást változatos könyezetekben is biztosíthatja. A
modulásri kódszervezés lehetővé teszi, hogy a robot funkcionalitását minél
könnyebben építhessük fel saját, vagy mások által készített csomagokban található
nodeok összekapcsolásával.

A node-ok belső működését befolyásolni node paraméterekkel tudjuk. A node
paraméterek alklamasak alkalmazásfüggő és kontextus specifikus információk
biztosítására, ezáltal a működés általánosabb implementációja válik lehetségessé
az adott node-on, amelyet így többfajta környezetben is felhasználhatunk.

Az node-ok összekapcsolását interfészeken keresztül tehetjük meg, amelyekhez a
ROS keretrendszer támogatást biztosít.

\subsubsection{Package}

A moduláris kialakításból adódóan szükség van egy olyan megoldásra, ami a
különböző node és funkciók menedzselésére és megosztására alkalmas. A ROS erre a
célra rendelkezik egy csomagkezelő funkcióval. A rendszer ezáltal kiegészíthető
adott funkcionalitást támogató csomagokkal, illetve saját csomagjaink
publikálására is van lehetőség. Egy ROS csomag azonban eltér egy hagyományos
Linux csomagtól, rendszerint ugyanis forráskódor, dokumentációt, és build
fileokat tartalmaz, futtatható binárisok helyett. 

A package általában rendelkezik egy vagy több konfigurációs file-al, amelyek a
package-hez tartozó metainformációt tárolják. Az egyik legfontosabb, a package
által biztosított entry pointok listája, amelyben a packageben tárolt szoftver
vagy szoftverek belépési pontjait defíniálhatjuk. Ílymódon egy csomag több
belépési pontal is rendelkezhet, ami lehetővé teszi egy adott funkcióhoz tartozó
több program egy package-en belüli tárolását.

\subsubsection{Interfészek}

A node-ok közti kommunikáció a jól működő ROS projektek kulcsa, a  kommunikáció
megvalósítását interfészek segítségével valósítjuk meg.

Az interfészek működésük szerint három típusba sorolhatóak: topic-ok, service-ek,
és actionok.  

\medskip

A \textbf{topic} egy publisher-subscriber modellt megvalósító interfész. A
topicra információt lehet küldeni, illetve fel lehet rá iratkozni. Amikor új
információ jelenik meg, azt minden feliratkozott node megkapja. Alkalmazására
tipikus példa egy szenzor értékeinek olvasása. Egy node olvashatja az érzékelőt,
de több node-nak lehet szüksége az új adatra, ilyen esetben topic alkalmazása
kézenfekvő megoldás.

\medskip

Az alkalmazásban felmerülhetnek olyan helyzetek, amikor kérdés-válasz jellegű
kommunikációra van szükség, például egy egyszeri műveletvégzés, amely
eredményéről visszajelzés érkezhet. Ilyen esetekben a \textbf{service} interfész
jelent megoldást. Minden service-hez tartozik egy server-node, amely a műveletet
végrehajtja, és a service-en keresztül visszajelzést ad.

\medskip

A hosszabban tartó feladatokban, mint például a vezérlési feladatok, az
\textbf{action} típusú interfészek lesznek segítségünkre. Az action összetett
interfész, amely egy folyamat elindítását, és folyamatos visszacsatolás
monitorozását végzi. Egy action működése két service és egy topic
felhasználásával megfeleltethető, ezeket goal-service-nek, feedback topic-nak és
result-service-nek nevezzük. A goal-service a feladat számára a cél
specifikálását végzi, amelyre a kiszolgáló node válaszol. A result-service-en
keresztül eredmény tájékoztatás kérhető, amely a feedback topic-on keresztül
csatolódik vissza a kérdező node felé. A feladat végeztével a result-service
nyugtázza a végrehajtott feladatot.

Erre a funkcionalitásra példa lehet egy léptetőmotor beállítása adott
állapotba. Action alkalmazásával a kívánt szög beállítható, majd a cél eléréséíg,
vagy esetleges megszakításig folyamatos pozícióinformáció áll a rendelkezésre.

\medskip

Az interfészeken zajló kommunikációnak egyértelműnek kell lennie mindegyik
kommunikáló fél számára, az információ típusa ezért kulcsfontosságú. A ROS
lehetőséget biztosít számunkra saját típus defíniálására, valamint package
formájában szintén számos típus érhető el.

Saját típusok alkalmazása mellett az interfészek használata rendkívül flexibilis
megoldást nyújt a node-ok közötti kommunikáció biztosítására. 

\section{A ROS driver}

A Linux image sikeres telepítését és a ROS rendszer tesztelését követően már csak
egy feladat maradt hátra, a robot hardverének integrálása a ROS környezetbe, és
teszt alkalmazás készítése. Az integrálásra a ROS környezetnek saját kiépített
megoldása van, amely a \verb|rosserial| csomagban elérhető. Ennek a kialakítása
azonban egy saját protokollhoz kötött, és integrálása több munkával járt volna,
mint egy saját illesztő megoldás készítése, amely a ROS rendszer számára a robot
funkcióit biztosítja.

\medskip

Az első feladat egy olyan ROS node, vagy node-ok elkészítése, amelyek a robot
hardverét megfelelően illesztik a ROS többi komponensével. A feladathoz
kiegészítő feladatként egy teszt előállítása is tartozik, amely a driver node
működését monitorozza, ezáltal teszteli működését.

\medskip

A második feladat a labirintusban való tájékozódás kialakítása, amely során a ROS
keretrendszerben egy olyan node architektúrát kell kialakítani amely az ismert
topológia bármely pontjáról bármely elérhető pontra el tud navigálni.

\subsection{A fejlesztés menete}

A ROS egy multiplatform keretrendszer így telepíthető ARM és x86 architektúrájú
rendszerekre egyaránt, valamint támogat Linux, Windows és Mac
operációsrendszereket is. A fejlesztést így egy asztali számítógépen is lehet
végezni, amelyet a kész package ``kiadását'' követően lehet telepíteni a
célplatformra. Ezzel a fejlesztési módszerrel sok idő megspórolható, és
könnyebben, ergonómikusabb környezetben lehet tesztelni a node-okat.

\medskip

A ROS fejlesztés a Yocto Projecthez hasonlóan igényel egy megfelelő shell
környezetet, amely környezet előállításához scripteket biztosít. A scripteket
sourceolva a környezet beállítódik, minek következményeként például a ROS
binárisai bekerülnek a \verb|PATH| környezeti változóba és közvetlenül
indíthatóak lesznek. A scriptek automatikus source-olásához a fejlesztői shell
init file-jában elvégezhetjük a sourceolást.

\subsubsection{Workspace}

A ROS nem csak a node-ok futtatására, és a rendszer kiépítésére biztosít
támogatást, hanem a ROS csomagok fejlesztésére is. A ROS disztribúcióban találunk
eszközöket a ROS csomagok fodítására és menedzselésére.

A ROS package-k fejlesztésének támogatására a keretrendszer két programot
biztosít. A packagek függőségeinek menedzselésére szolgál a \verb|rosdep|
program. Ennek a programnak a feladata a szükséges csomagok letöltése és
telepítése. A packagek buildelésére a \verb|colcon| nevű program szükséges, amely
a package-ek buildelését menedzseli.

A ROS csomagok fejlesztése egy workspace-en belül történik. A workspace
lényegében egy mappa, amelyben a ROS package-eket tároljuk, forrásaikkal, build
artifactjeikkel, és egyéb szükségés file-jaikkal együtt.  A mappa létrehozása
után létre kell hozni egy almappát a csomagok számára, amelynek legyen a neve
\verb|src|.  A colcon a workspace maradék részét automatikusan létrehozza a
\verb|build| parancs kiadásával, de fontos hogy a parancsot a workspace root
mappájában adjuk ki.

A colcon létrehoz egy \verb|build|, \verb|install|, és \verb|log| mappát. A
\verb|build| mappában találjuk a lefordított csomagokat, az \verb|install| mappa
egy lokális installációt tartalmaz a fordítás kimeneteiből, a \verb|log| mappa
pedig a fordítás során keletkezett naplófileokat tartalmazza. Az install mappa
része egy \verb|local_setup.sh| script is, amelyel a lefordított packageket mint
overlay tudjuk használni. A script source-olásával a csomagok elérhetővé válnak a
ROS rendszer számára.

\subsubsection{Overlayek}

A ROS réteges kialakításban működik, az alap ROS2 installációra épülő rétegeket
overlayeknek nevezzük. A ROS2 az alaptelepítést underlay-nek, a felhasználó által
létrehozott rétegeket overlaynek nevezi.

Az overlayek kezeléséhez a rendszer installációs scripteket biztosít. Az
overlayek felhasználásával egy rendszeren található ROS2 installáció további
csomagokkal bővíthető az eredeti installáció módosítása nélkül. Az overlayek
felhasználása különösen hasznos fejlesztés során, amikor egy stabil rendszert
kell összerakni.

\subsubsection{Packagek}

A packagek létrehozása lehetséges manuálisan is, de a ROS saját parancsal is
rendelkezik az új package létrehozására, amely gyorsabb és biztosabb megoldás. A
létrehozáskor paraméterben megadhatjuk a csomag nevét, valamint a használt build
rendszert, amely alapértelmezés szerint vagy python vagy cmake.

Az új csomag létrehozásánál fontos, hogy a parancsot a workspace \verb|src|
almappájában adjuk ki, amennyiben nem így járunk el, a build folyamat
összeakadhat. A létrehozott package mappájában csak a package szempontjából
fontos fileok találhatóak meg, minden build kimenetet a colcon a build mappában
hoz létre, így a mappa könnyedén verziókövethető, nincs szükség \verb|.gitignore|
szerkesztésre.

A package konfigurációja eltér a választott buildrendszer függvényében. Minden
csomag rendelkezik a \verb|package.xml| konfigurációs file-al, amelyben
defíniálva van a csomag neve, verziója, és maintainere, a csomaghoz tartozó
licensz, a használt buildrendszer, és a csomag függőségei.

CMake alapú csomagok esetében minden egyéb konfiguráció a \verb|CMakeListe.txt|
file-ban található, úgy mint például az előállítandó binárisok, valamint a
belépési pontok.

Python alapú csomag esetében azonban még két konfigurációs file létezik, a
\verb|setup.cfg|, és a \verb|setup.py|. A package számára fontos konfigurációk a
\verb|setup.py| fileban vannak, ahol a \verb|package.xml| mintájára megtalálható
néhány metainformáció, valamint a csomag által biztosított belépési pontok.

\subsection{A driver node}

A feladat első részében a driver node létrehozásával foglalkoztam. A node
fejlesztéséhez választott nyelv a python volt, lévén a feladat nem igényelte a
C++ nyelv használatát. A Python nyelv mellett további érv, hogy a fejlesztés
sokkal gyorsabban végezhető ezen a nyelven, mint C++-ban, valamint python nyelven
érhető el rengeteg útmutató, és dokumentáció, amely a kezdő fejlesztőknek segít
megismerkedni a környezettel.

\subsubsection{Általános kialakítás}

A drivert úgy terveztem, hogy egyetlen node által legyen biztosítva a
kapcsolat. Ennek az oka, hogy a specifikált protokolban megkövetelt szigorú
sorrendiség könnyebben tartható, ha egyetlen komponensnek van hozzáférése a soros
porthoz.

A driver node egyetlen soros port illesztésén és a megfelelő protokoll
implementálásán alapul. A Raspberry Pi esetében az usart interface egy device
file formájában érhető el, ezért kezelése könnyen megvalósítható. A port
meghajtására a Python \verb|serial| library-jét használtam, amely az USART
kezelését végezte. A fejlesztés és deployment környezet közötti eltérés miatt,
valamint az interfész node minél általánosabb kialakítása végett node
paraméterben határoztam meg a használt baud rate-et valamint a felhasznált device
file-t.

\subsubsection{Szenzor és sebesség olvasás}
A node-ot megvalósító osztály tartalmaz egy referenciát a device file-ra, amelyen
keresztül a firmware modullal kommunikál. A firmware irányából két mérhető érték
kérdezhető le, ezek a szenzor-értékek, valamint a sebesség. Mindkét értékhez
külön topic tartozik, amelyeket az illesztő node hoz létre.

A rendszeres lekérdezés egy belső szoftveres timer felhasználásával van időzítve,
amelyet a ROS keretrendszer biztosít. A timer konfigurálható időközönként jelez,
amely jelzéshez a node egy callback függvényt implementál. A callback függvényben
történik a lekérdezés, amely során a node az információkat lekéri a firmwaretől,
feldolgozza a karakteres formában kapott információt, majd publikálja a megfelelő
topic-okba. A robot szenzorai, és enkódereiből számolt sebesség ezáltal elérhető
a ROS rendszer többi komponensének számára.

\subsubsection{Motorvezérlés}

A robot motorjának hardveres meghibásodása miatt nem volt módom
implementálni a haladás megvalósítását. Ezutóbbi funkció a robot feladatának
ellátásához kritikus, így egy elméleti megvalósítást mutatok be.

A robot motorjainak vezérlésére legmegfelelőbb interfész az action.
Az action céljaként specifikálható adott távolság, amelyet a robotnak meg kell
tennie, vagy adott szög, amelyet a robotnak fordulnia kell. A sebességmérés
lehetővé teszi, hogy a motorok sebességéből és az eltelt időből a node kiszámítsa
az adott idő alatt megtett távolságot, amelyet ezek után akkumulálva egy
visszacsatolásban publikáljon.

A motorvezérlés ily módon való megvalósításával a robot egyszerű mozgási
parancsokat képes végrehajtani. Ezt a funkciót más feladatot ellátó node-ok
könnyen használhatják a feladat elvégzésére.

\subsection{Echo-test}

A csomagnak része egy tesztelés céljára készített node, amely a driver node által
létrehozott topic-okra feliratkozva monitorozza a driver működését. A cél
rendszeren elindítva sikeresen teszteltem a driver szenzorolvasó
funkcionalitását, valamint egy motor manuális forgatásával a sebességmérést is.

\subsection{Típusok}

Az interfészek megvalósítása során szükséges, hogy pontosan defíniáljuk az adott
interfészeken áthaladó üzenetek típusait. A ROS keretrendszer biztosít számunkra
saját platformfüggetlen interfésztípus defíniálására lehetőséget, cmake típusú
package formájában. A projektnek ezért része egy csomag, amely az interfészekben
használt típusokat defíniálja. 

A saját típusok defíniálására a ROS rendszerben külön file-ok szolgálnak,
amelyekben speciális szintaxis használatával van lehetőség egyedi típus
készítésére. A használt típusok formátuma azonban erősen függ a felhasznált
interfésztől, például hogy egyirányú a kommunikáció mint egy topic esetében, vagy
kétirányú ahol válaszra is van szükség, mint egy service. A típus meghatározása
során megadhatjuk az adatok típusát, amelyek lehetnek beépített típusok, vagy más
csomagok által biztosított egyedi típusok, valamint megadhatjuk az adott
adatmezőt azonosító elnevezést is. Olyan interfészek esetén, ahol több szegmensre
van szükség, mint például a service, ott a szegmenseket három kötőjellel
elválasztva határozhatjuk meg.

A robothoz készült interfész csomag két típust biztosít a rendszer számára, a
Distance, és Velocity típusokat. A Distance típus négy, előjel nélküli tizenhat
bites egészt tartalmaz, a négy iránynak megfelelően. A Velocity a motorok
elhelyezkedése szerint két előjeles tizenhat bites egész számot tartalmaz.

A csomag \verb|CMakeLists.txt| file-jában a ROS rendszer által szolgáltatott
CMake kiegészítést felhasználva megadható azon fileok listája, amelyekből a
típusok generálásra kerülhetnek.


\section{A labirintus probléma}

A projekt utolsó, része a teszt alkalmazás fejlesztése, amely a tényleges
labirintusban való navigálást biztosítja. A motorokból adódó hiba okán a
motorvezérlés és a ráépülő feladatok nem készültek el. A következő fejezetben
bemutatom a feladat értelmezését, és megoldási tervet valamint fejlesztési
lehetőséget vázolok fel az utolsó feladathoz.

\subsection{A Labirintus}

A feladat során specifikálva van, hogy a labirintus topológiája ismert. A
megadásának módja valamint egyéb jellemzői azonban nem. A környezet mivolta
erősen befolyásolja a végrehajtást, így a következőkben két lehetséges modellt
vázolok fel:

\subsubsection{Az egyszerű eset}

Egy ideális labirintusban minden út egyenlő széles, amelyben a robot képes
végighaladni valamint akár teljes fordulatot is tenni. A fordulópontok szabályos
derékszögű kanyarok, és minden pontból minden pontba el lehet jutni.

\subsubsection{A komplex eset}

Egy összetettebb eset nagyobb esélyel modellezi a valós problémát, így nagyobb
komplexitással is bír. Komplex esetben a labirintus falai nem minden esetben 


\subsection{Algoritmusok}
\subsubsection{Az egyszerű eset}

\medskip

Az egyszerű labirintus könnyedén ábrázolható egy gráf segítségével, amelyben a
gráf csúcsai a lehetséges fordulópontok. A gráf éleiben kódolható a navigációhoz
szükséges két információ. Elsőként a fordulat iránya amely lehet  jobb, bal és
előre, majd a következő csomópontíg megteendő távolság.

Egy ilyen gráfban egy legrövidebb út kereséssel meghatározható, hogy milyen
sorrendben kell az éleken végíghaladni, ami aztán egyértelműen defíniál egy
haladás-fordulás sorrendet a robot számára. Ennek a végrehajtásával a navigáció 
elméletben megoldható.

\subsection{Kiindulópont}

\subsection{A haladás implementációja}

%% folyamatos szabályozás kell mert minden esetben a
folyt köv
