\chapter{Architektúra}

\section{A robot topológiája}

A robot tervezése sok kisebb feladatból és megtervezendő egységből, alegységből
áll. A tervezés során célravezető megközelítésnek tartottam a nagyobb egységek
megtervezésétől a kisebb alegységek megtervezéséíg haladni. Így a specifikációtól
indulva az aktuális feladatot alfeladatokra bontva volt lehetőségem kidolgozni a
robot részleteit annélkül, hogy a részletekben nagyon elvesznék.

Ezt a megközelítést követve a specifikáció alapján meghatároztam a modulokat,
amiket a robot működéséhez létre kell hozni. Ezt követően meghatároztam az
interfészeket, amin a modulok egymással kapcsolatban állnak, maje ezekből egy diagrammot
készítettem, amelyen a rendszer áttekinthető. 

\missingfigure{blokkdiragram a robot kialakításáról}

\subsection{A Differenciális robot}

A tervezés első fázisa a robot kialakításának kiválasztása volt. Az autonóm
robotok számos elrendezésben megtervezhetők, amely a robot alkalmazásától függően
lehet komplexebb vagy egyszerűbb. Bizonyos környezetek robosztusabb kialakítást
igényelnek, más környezetek nagyobb mozgékonyságot. A projekt szempontjából a
differenciális elrendezést tartottam a legcélravezetőbbnek.

Az autonóm robotok között a differenciális robot egy relative könnyen
megvalósítható konstrukció, így a projekt fokuszában egy ilyen konstrukció
mellett maradtam. Ennek a robot típusnak összesen két motor áll rendelkezésére
hogy helyzetét és orientációját változtassa, ezáltal könnyebben megvalósítható
mint komplexebb meghajtással rendelkező társai, viszont mozgékony és sok
lehetőséget tartogató kialakítás.

\missingfigure{Ábra a differenciális robotról}

Klasszikus elrendezése, hogy a két motort a hosszanti tengellyel merőlegesen kell
elhelyezni, ezálal a a robot sebessége a motorok szögsebességéből valamint a
kerékátmérőből számítható. A robot orientációját a két motor különböző nagyságban
és/vagy különböző sebességgel történő meghajtásával lehet vezérelni.

A kialakítás további előnyeihez tartozik, hogy a robot vezérléséből csak relative
kevés erőforrást kell a mozgásra allokálni, hiszen pusztán csak a két motorhoz
tartozó szabályozó és irányító algoritmusokat kell futtatni, ugyanakkor a robot
sík, vagy közel sík terepen minden különösebb nehézség nélkül képes navigálni,
ezáltal sokféle alkalmazásban ideális választás lehet.

\subsection{Modulok és feladatkörök}

A robot megalkotását modulok megtervezésére és realizációjára alapoztam, ennek a
megközelítésnek a tervezésen kívül a végtermék minőségében is jelentős hatásai
voltak. A kész robot ugyanis különálló, de jól meghatározott interfésszel
rendelkező modulok összességéből áll össze, így ha egy alkatrész meghibásodik,
vagy a minősége nem megfelelő, az eszközt könnyebben lehet javítani, vagy a hibás
modult cserélni. Ezen felül, amennyiben új elvárás merülne fel, úgy kevés
alkatrész cseréjével, vagy újak implementálásával könnyen alkalmassá tehető a
robot, egyéb feladatok ellátására is.

A projekt ennek a megközelítésnek köszönhetően könnyen fejleszthető, amennyiben
igény merülne fel adott modulok letisztázására, vagy funkcióköreik bővítésére. A
szóban forgó modul ugyanis, ameddig betartja a számára meghatározott interfészt,
minden különösebb nehézség nélkül fejleszthető és módosítható úgy, hogy a robot
működőképes marad.

\missingfigure{részletes blokkdiagramm, kommunikációs csatornákkal, és
  protokollokkal}




\todo{Ide jön a robot leírása, hogy hogyan akarom kialakítani milyen modulok
  lesznek bnenne meg minden ilyen shut}


\section{A terv értékelése}

\subsection{Pozitívumok}

A projekt moduláris megközelítéséből nagyon sokat profitáltam, a feladatok
ugyanis bizonyos megkötésekkel párhuzamosíthatóak voltak, ami a fejlesztést
nagyban gyorsította és tette kényelmesebbé. Egy késleltetett rendelés például nem
akasztotta meg teljesen a munkát, csak egy részmodult késleltetett.

Az adott modulok tesztelése szintén sokkal könnyebbnek bizonyult úgy, hogy egy
előre defíniált interfészt vagy protokollt tudtam követni. Így egy feladatot
könnyebben lehetett elvégezettnek tekinteni ha a tesztelésen átesett, nem kellett
az egész konstrukciónak elkészülnie a részeredményekhez. 

\subsection{Hátrányok}

Az előre defíniált interfészek jelentettek nehézséget is. Ha egy modul
fejlesztésében nem várt nehézséget okoz egy bizonyos interfész betartása, akkor
az sok fejlesztési idő kiesését tudja jelenteni.

A projekt fejlesztése nagyon erősen függ olyan döntésektől amikben a fejlesztést
megelőzően kevés tapasztalattal rendelkeztem, hogy hatékony interfészt határozzak
meg. Ilyen esetben vagy a már meglévő modulokhoz alkalmazkodva több idő és
energia ráfordítással fejlesztettem le a modult, vagy módosítottam az interfészt,
amennyiben nem, vagy csak nagyon kevés modul függött az adott interfésztől. Az
utólagos interfészmódosítások azonban nagy bizonytalansággal járhatnak, és a
legtöbb esetben érdemes azokat elkerülni.

\subsection{Fejlesztési lehetőségek}


\todo{Ide irunk minden olyat ami nem volt eléggé átgondolva a robot
  megalkotásakor: firmware, protokoll, táp, hardver, stb.}


\section{A mechanikus váz}

bevezető

\subsection{Anyaghasználat}

fa, csavarok, távtartók, méretek, furatok,

\subsection{Az elkészült váz}

értékelés
