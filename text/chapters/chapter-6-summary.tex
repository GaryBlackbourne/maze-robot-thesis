%----------------------------------------------------------------------------
\chapter{Összefoglalás}
%----------------------------------------------------------------------------

A projekt célkitűzése beható vizsgálat volt egy robot megalkotásának lehetőségét
illetőleg, Yocto Project és ebben integrált ROS2 keretrendszerek segítségével. A
feladatban a két rendszer megismerése és áttekintése után, a fejlesztési és
fordítási lehetőségek felmérése is cél volt. A rendszerek demonstrálására a
feladat részét képezte egy robot megépítése és dokumentációja is.

\medskip

A projekt során egy teljes robot rendszert készítettem. A robot tervezését
feladatokra és modulokra bontva, részletesen ismertettem.  Munkám során minden
modul esetén bemutattam a fennálló problémat, a tervezés lépéseit, a mérnöki
döntések mögött álló indokokat és érveket, végül szót ejtettem az adott
témakörökben tapasztalt nehézségekről is.

A munkám elején áttekintettem a robot kialakítását, és részegységekre bontottam a
felépítését, hogy így átlátható rendszerben történhessen a fejlesztés.

Ezt követően bemutattam a robot hardveres elemeit amelyek a projekt alapjaként
szolgálnak. Röviden kitértem ezek képességeire, kialakításukra, és szerepükre az
egész rendszer szempontjából.

A következő lépésben részletesen bemutattam a robot vezérléséhez szükséges
firmwaret. Ismertettem fejlesztésének módját és a keresztfordító környezetet. A
fejezetben bemutattam a kész firmware kialakítását, szemléltettem az egyes taskok
szerepét, az időkritikus feladatok esetében bemutattam az optimalizáció hatását,
és munkámat mérési eredményekkel támasztottam alá. 

A következő fejezet során a Yocto Project ismertetésével foglalkoztam, és
bemutattam a beágyazott Linux image készítnésének módját a projekt példájának
alapján. Ismertettem a projektben használt image tulajdonságait, valamint
bemutattam a ROS2 rendszer integrálását is. A fejezetben kitértem saját ROS2
package-k telepítésére, a Yocto által használt receptek segítségével.

Munkám lezárásában a ROS2 keretrendszert mutattam be, amelynek során sikeresen
integráltam a rendelkezésre álló robot komponenseit, és sikeres tesztet végeztem
a roboton futtatott ROS2 alkalmazás segítségével.

\medskip

A projekten való munkával hasznos tapasztalatokat szereztem a beágyazott
rendszerek fejlesztésének minden területén, különös tekintettel a beágyazott
szoftverfejlesztésre. A projekt során megismertem egy komplex rendszer
megtervezésével és megalkotásával járó kihívásokat, és tapasztalatot gyűjtöttem a
különböző területek egymásra hatásáról egy projekten belül.

\section{Fejlesztési lehetőségek}

Az elkészült hardver számos irányba fejleszthető tovább. Moduláris kialakításának
köszönhetően a fejlesztés irányultsága is eltérő lehet. A robot alacsony szintű
moduljaiban több ponton találunk hardveres javításra és letisztázásra
lehetőséget, akár Quality-of-Life jellegű, akár új funkcionalitás implementálását
célzó módosítások formájában.

\medskip

A kész robot motoros vezérlésének tesztelése az első és legfontosabb lépés a
projekt fejleszthetősége szempontjából. Ennek a funkciónak az implementálása a
robot funkcióit teljeskörűvé tenné, és további szoftveres fejlesztések számára
biztosítana alapot.

Az akkumulátoros üzemű rendszerek számára hatalmas előny ha az akkumulátor
állapotáról visszacsatolás érkezik. A projekt fejlesztésére jó lehetőség az
akkumulátormenedzsment rendszer újragondolása, és egy komplexebb tápáramkör
implementálása, amely kapcsolatban áll a robot többi részével.

\medskip

A firmware fejlesztés szempontjából a motorokra épül egy fontos fejlesztési
lehetőség, a PID szabályzás. Ennek implementálása a motorvezérlésben nagyban
javítaná a firmware minőségét. A firmwareben másik fejlesztési lehetőség egy
robosztus kommunikációs protokoll kialakítása.

\medskip

A Linux image-et érintően a robot saját hálózatának kiépítése és alapvető
autentikáció implementálása hozzájárulna a robot könnyebb kezelhetőségéhez. A
firmware keresztfordításának Yocto integrációjával a robot előállítását lehetne
tehermentesíteni.

\medskip

A ROS rendszert tekintve rengeteg fejlesztési lehetőség adódik. A felsorolt
lehetőségek mind alapot adnak a ROS rendszer bővítésére, lehetővé téve, hogy a
robot fő funkciói számos feladatban és környezetben alkalmazhatók legyenek.

