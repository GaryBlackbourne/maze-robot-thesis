%----------------------------------------------------------------------------
\chapter{Előzmények}
%----------------------------------------------------------------------------

% feladatértelmezés legyen, cél meghatározás,
% szükséges eszközök (libek alkatrészek) felmérése
% ezek rövid bemutatása

% legyen tervezés szegmens indoklásokkal és realizáció fejezet amiben a
% megvalósítást és a tapasztalatokat problémákat írom le.

\section{A részfeladatok}

A diplomatervezés feladat keretein belül egy autonóm robot megtervezése és
megépítése volt a cél. Ez a feladat egy komplex tervezési feladat, amely több fő
részfeladatra bontható. A munkám során ezeket a részeket, mint kisebb önálló
részfeladatokat, egymás után oldottam meg, kisebb-nagyobb átfedésekkel. Ezeket a
részfeladatokat az alábbi módon határoztam meg:

\begin{itemize}
\item Hardver
  \begin{itemize}
  \item Mechanikai kialakítás megtervezése és kialakítása
  \item Szenzor és motor kiválasztása
  \item Energiaellátás megtervezése és kialakítása
  \item Motor és szenzorvezérlő áramkör megtervezése és kialakítása
  \item Magasabb szintű döntéshozó egység választás
  \end{itemize}
\item{Firmware}
  \begin{itemize}
  \item Motor és szenzorvezérlő áramkör firmware megtervezése és kialakítása
  \end{itemize}
\item{Software}
  \begin{itemize}
  \item Beágyazott linux rendszer készítése
  \item ROS2 keretrendszer telepítése
  \item A robot hardver integrálása a ROS2 keretrendszerbe
  \item Demo alkalmazás elkészítése amelyben a robot labirintusban halad.
  \end{itemize}
\end{itemize}

\section{Az előzmények}

A feladat egyik központi részének alapját a robot alkotó elektronikai komponensek
jelentették. A feladathoz felhasználtam a korábbi, önálló laboratóriumi projekt
kereteiben elkészített áramköri paneleket, amelyek egy témában a diplomatervhez
nagyon hasonló alkalmazáshoz készültek. Ezek biztosították a hardveres alapot a
robot kialakítása során.  A robot koncepcióját bemutató következő fejezetben
részletesebben kitérek a panelek felhasználására és topológiájára, a jelen
fejezetben pusztán az elektronikai igényekről, specifikációkról valamint az
alkatrészek megválasztásáról lesz szó.

\subsection{Motor és szenzor választás}

A robot két fő ponton kapcsolódik a környezetéhez amiben a feladatát ellátja: a
szenzoraival, és a motorjaival. Ezen alkatrészek kiválasztása a tervezési fázis
elején történt, és meghatározó szerepű volt.

\medskip

A szenzorok biztosítják az információszerzés lehetőségét a robot számára a
külvilág felől, így a pontos feladatvégrehajtáshoz megfelelő szenzor kiválasztása
elengedhetetlen. Minden feladathoz szükséges a végrehajtáshoz szükséges
információk kinyerése a környezetből, ezeknek az információknak a tipusa eltérő
lehet. A navigációhoz nagyon gyakran alkalmazunk távolságmérést aminek több módja
van, mint például a radar, a szonár, vagy esetleg a lidar. A robot esetében az
olyan szenzorok kiválasztására volt szükség, amelyekkel pontos távolságméréssel
falak relatív pozícióját lehetett meghatározni.

Rövid kutatás után az STMicroelectronics által gyártott VL53L1 szenzorchipekre
esett a választásom. Ezek a chipek voltak a leggyakoribb keresési találatok, és
jó visszajelzéseket találtam róluk. Bár a kommunikáció a chippel I2C buszon
zajlik, a szenzorok komplexitása és vezérlésük bonyolultsága miatt, vezérlésükhöz
a gyártó külön driver csomagot biztosít. A gyártó nem csak integrált áramkör
formájában forgalmazza ezeket az alkatrészeket, hanem VL53L1-SATEL néven, mint
modul is kaphatóak. Ebben a kiszerelésben a táp- és földlábak, az I2C busz
valamint két fontos láb, ki vannak vezetve a panel oldalán található tüskesorra.
A panel 3.3V feszültséget igényel, amiből a szenzor IC számára előállít egy
alacsonyabb feszültségszintet. Ez a projekt szempontjából ideális, mert nincs
szükség még egy tápfeszültségszint előállítására.

\missingfigure{ábra a szenzorokról}

\medskip

Az autonóm robot helyzetváltoztatása motorok segítségével történik. A motorok
meghatározásánál a feladat végrehajtásához szükséges teljesítmény és
forgatónyomaték igény, valamint a robot által leadható teljesítmény a két
legfontosabb paraméter. Fontos szerepet játszik továbbá a motor mehajtásának
módja, amely a kisebb robotok esetében szinte kizárólag DC motorok felhasználását
jelenti. A robot tervezése során fontos kiemelni, hogy a motor egyben érzékelő
is, ahhoz, hogy pontos beavatkozást végezhessünk, a motorok sebességének mérésére
is szükség van. A motorok sebességének mérésére leggyakrabban enkódereket
használunk, amiket néhány gyártó motorjaiban beépítve találunk, más esetekben
magunknak kell felszerelnünk azokat a motor tengelyére.

A projektben 6V DC meghajtású, szénkefés motorokat választottam, mert az ezekhez
tartozó feszültségszintet könnyen elő tudtam állítani, és teljesítményben
megfeleltek az alkalmazás számára. A motorok típusa N20E villanymotor, amelyet
könnyen be tudtam szerezni. Az alkatrészt kisméretű áttétellel szerelték fel, ami
a maximális elérhető fordulatszámot növelte a maximális nyomaték beáldozásával. A
motoron megtalálható volt továbbá egy inkrementális enkóder, amelyhez panelre
forrasztott tüskesoron keresztül lehetett áramkört illeszteni. Ezek a tárcsák \(2
* 7\) beosztást tartalmaztak.

\missingfigure{ábra a motorokról}

\subsection{A tápellátás}


\subsection{A vezérlő elektronika}


\section{Az elvégzendő feladatok}

\subsection{Mechanikai tervezés és kialakítás}

A projekt során a robot prototípus modelljét készítettem el, így a mechanikai
tervezésre kisebb hangsúlyt fektettem. Ennek a részfeladatnak a legszemléletesebb
része az áramkörök topológiájára és a mechanikai paraméterekre gyakorolt hatás
volt. A következő fejezetben röviden bemutatom a robot mechanikai kialakítását.

\subsection{A robot magasabb szintű vezérlése}

A magasabb szintű logikát végző processzorhoz, egy SBC\footnote{Single Board
Computer}-t használtam. Ezek a panelek egy nagyteljesítményű processzort, vagy
SoC\footnote{System On a Chip}-ot hordoznak, amelyet így könnyebben
intergrálhattam a robot rendszerébe. A választott SBC egy raspberry pi 4 B
modell, amelyet széleskörűen alkalmaznak hobbielektronikai megoldásokban, és
ipari felhasználásra is van példa. Ennek a boardnak könnyen hozzáférhető
dokumentációja van és rendelkeztem már az eszközt illető tapasztalattal.

\section{A projekt célja}

Cél megfogalmazása








\medskip
%% a robot kialakítása
Az autonóm robotok között egy relative könnyen megvalósítható konstrukció a
differenciális robot, így a projekt fő fokuszában egy ilyen konstrukció  mellett
maradtam. Ennek a robot típusnak összesen két motor áll rendelkezésére hogy
helyzetét és orientációját változtassa, ezáltal  könnyebben megvalósítható mint
komplexem meghajtással rendelkező társai, viszont mozgékony és sok lehetőséget
tartogató kialakítás. Klasszikus elrendezése, hogy a két motort a hosszanti
tengellyel merőlegesen kell elhelyezni, ezálal a  a robot sebessége a motorok
szögsebességéből valamint a kerékátmérőből számítható. A robot orientációját a
két motor különböző nagyságban és/vagy különböző sebességgel történő
meghajtásával lehet vezérelni. A kialakítás további előnyeihez tartozik, hogy a
robot vezérléséből csak relative kevés erőforrást kell a mozgásra allokálni,
hiszen pusztán csak a két motorhoz tartozó szabályozó és irányító algoritmusokat
kell futtatni, ugyanakkor a robot sík, vagy közel sík terepen minden különösebb
nehézség nélkül képes navigálni, ezáltal sokféle alkalmazásban ideális választás
lehet.

A projekt egyik fő célja az volt hogy minél átfogóbb ismereteket szerezzek az
egyes témakörökben amiket a robot tervezése során érintek, ezeket a dolgozat
további részeiben rendre dokumentálom.

A robot tervezése során 9 fő feladatot határoztam meg:

A robot kialakítása során fontos szempont volt a modularitás. Így ha egy
alkatrész meghibásodik, vagy a minősége nem megfelelő, az eszközt könnyebben
lehet frissíteni, fejleszteni. Ezen felül, amennyiben új elvárás merülne fel,
úgy kevés alkatrész cseréjével, vagy újak implementálásával könnyen alkalmassá
tehető a robot, egyéb feladatok ellátására is.

A feladatok könnyebb strukturálása valamint a rendszer pontosabb átláthatósága
érdekében blokkdiagramot is készítettem:

\missingfigure{ide jön majd a blokkdiagram a robotról}

Az önálló laboratórium során több problémába is ütköztem, melyek közül az egyik
legfontosabb a chiphiány, amely jelentősen megnehezítette a  megfelelő
alkatrészek megtalálását és ezzel jelentősen lelassította a tervezési fázist is.

Az önálló laboratórium keretei között végül sikeresen elkészült a tápellátásért
felelős panel, és a motorvezérlést biztosító panel. Döntés született a
felhasználandó motorok, szenzorok, és egyéb alkatrészek ügyében, valamint
elkészült egy kezdetleges firmware amelyel a mikrokontroller perifériáit
tesztelte és egy hozzá tartozó build környezet, amely lehetővé tette számomra,
hogy minden munkát a saját számítógépemen egy linuxos környezetben végezzek.

A projekt ebben az állapotban ért az önálló laboratórium végére, így a hátralévő
feladatok, valamint az újonnan defíniált célok már a diplomaterv feladataiba
tartoznak.

%----------------------------------------------------------------------------
\section{Alkatrészválasztás}
%----------------------------------------------------------------------------

A megfelelő alkatrészek kiválasztása kritikus feladat volt a tervezési szakasz
legelején.

Első lépésben a robot hozzávetőleges tömege és kialakítása alapján választottam
motorokat, amelyek a fő beavatkozó szervek. A beépített enkóderek és a megfelelő
tápfeszültség alapján az n20e motorok mellett döntöttem. Ezek több
konfigurációban is kaphatóak voltak, és inkrementális enkóderrel szerelték fel
őket, így a sebességmérés is lehetővé vált ezeknek a motoroknak a használatakor.
A kereskedő weboldalán a motorokhoz megfelelő felfüggesztést, és tengelyre
szerelhető kereket is találtam, ami a mechanikai szereléseknél nagy segítség
volt. Ezek a motorok kefés DC motorok amelyeket gyakran használnak
hobbielektronikában. A motorok kapocsfeszültsége 6V, valamint a beépített
enkóderek működéséhez 3.3V-tól 5V-íg tetszőlegesen megválasztható feszültségszint
tartozik.

A következő lépésben a távolságérzékeléssel foglaloztam. A távolságérzékelőknek
több fajtája létezik, működési elveikből kifolyólag. A projekthez én az
STMicroelectronics VL53L1 érzékelőjét választottam, amit előzetes kutatásaim
és ajánlások alapján választottam. Ez a szenzor egy LIDAR, lézer alapú
távolságmérő berendezés. A szenzor egyik nagy előnye volt az I2C interfész, amin
a szenzorral kommunikálni lehet. Erős érv volt továbbá a VL53L1-SATEL board, ami
egy kis méretű development board amelyre a szenzor fel van ültetve. A board
egyik felén standard tüskesorhoz tartozó furatok vannak, amelyekre a szenzor 
fontos lábai ki vannak vezetve. A board ennek köszönhetően könnyen felszerelhető,
nem kell hozzá külön panelt tervezni, valamint a tápellátást is a SATEL boardra
lehet bízni. Ezzel a modullal a szenzor könnyen használható 3.3V
tápfeszültségről.

A projekt legelejétől fogva fontos lépés volt, hogy egy beágyazott linux alapú
vezérlő tudjon a magas szintű feladatokkal foglalkozni, hogy a firmware
komplexitása lecsökkenhessen, valamint a rendszer megőrizze modularitását. A
legelterjedtebb megoldás erre a célra a raspberry pi, amely egy ARM alapú
SBC\footnote{Single Board Computer}. Ennek az eszköznek hatalmas népszerűsége
van, valamint rendkívül jól dokumentált, így alkalmasnak éreztem erre a
feladatra.

A fenti specifikációkkal már el tudtam kezdeni a robot topológiájának
megtervezését, valamint a táp, és a vezérlőpanelek megtervezését.

A következő fejezetben bemutatom a robot tápellátásának teljes tervét az
elkészült táp panelt, valamint a vezérlőpanelt.

