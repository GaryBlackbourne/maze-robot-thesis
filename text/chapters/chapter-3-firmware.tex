%----------------------------------------------------------------------------
\chapter{Firmware}
%----------------------------------------------------------------------------




\missingfigure{drawios blokkdiagram a vezérlésről}

A mikrovezérlő firmware megírása során törekedtem a platformfüggetlen, eönnyen
áttekinthető kód készítésére. A teljesítmény maximalizálása és a kódméret
kordában tartása végett nem szerepel a projektben HAL\footnote{HAL: Hardware
Abstraction Layer} library kód. Minimális generált kóddal dolgoztam, amely
kimerül a linkerscript, valamint a startup fájlokban, ezeket a gyártó STM32CubeMX
nevű szoftverével hoztam létre. A vezérlő felkonfigurálásához, és perifériáinak
eléréséhez a CMSIS\footnote{CMSIS: Cortex Microcontroller Software Interface
Standard, az ARM által megkövetelt szabványos nevezéktan és támogatás minden
Cortex típusú magot tartalmazó mikrovezérlőre} által szolgáltatott
regiszterdefiníciókat használtam. A firmware megírása során arően támaszkodtam az
STM32F103 \todo{Ez egy hivatkozás az urlre}Reference Manualjára.

Ennek a firmware verziónak a képességei kimerültek egy USART periféria
használatában, a timer periféria pwm konfigurációjában, valamint a FreeRTOS
beágyazott operációsrendszer elindításában.

Ez a megközelítés nagyban segített a mikrovezérlő belső működésének pontos
megértésében, ellenben a diplomatervi munka során a fejlesztés gyorsítása
érdekében áttértem a HAL driverek használatára is, erről a későbbi fejezetekben
lesz szó.

Az önálló laboratórium alatt írt firmware végül nem lett közvetlenül hasznomra,
így teljes egészében cserélnem kellett.

Az önálló laboratóriumi munkám ennyit segített a kész produktumhoz, a következő
feladatokat már a diplomatervezés keretein belül hajtottam végre.
