%----------------------------------------------------------------------------
\chapter{Táp és vezérlő panelek}
%----------------------------------------------------------------------------

Az előző fejezetben a projekt motivációjáról, valamint előzményeiről írtam,
ebben a fejezetben bemutatom a tápellátó rendszert és a hozzá tartozó panelt,
valamint a vezérlő panelt, amely a robot motorjainak, és szenzorjainak
vezérléséért felelős.


%----------------------------------------------------------------------------
\section{A tápellátás}
%----------------------------------------------------------------------------

A robot t


%----------------------------------------------------------------------------
\section{A táp áramkör}
%----------------------------------------------------------------------------


A tápellátó áramkör feladata a rendszer minden komponensét energiával ellátni.
Ehhez három kimeneti feszültségszintre van szükség: \textbf{3.3V}, \textbf{5V},
\textbf{6V}. A teljes táp rendszer teljesítménye \textbf{30W} amely két részben
oszlik meg. \textbf{15W} jut a \textbf{6V} kimenetre és \textbf{15W} jut a
\textbf{3.3V} és \textbf{5V} kimenetekre. A feszültségszinteket egyesével a
szükséges komponensek igényei szerint határoztam meg. Az energiaforrás
tekintetében a projekt céljához a legmegfelelőbb konstrukciónak az akkumulátoros
üzemet tartom. Ez a robotot mozgásában nem akadályozza, valamint könnyen
cserélhető és skálázható amely a továbbfejlesztések és kiegészítések
szempontjából fontos.

A robot energiaellátása az alábbi alkatrészekből áll:
\begin{itemize}
\item akkumulátor blokk (3 * 18650)
\item BMS modul
\item kapcsoló üzemű táp
\end{itemize}

A BMS\footnote{BMS: Battery Management System} egy olyan modul amely a Lítium-ion
cellák vezérléséért felel, amely komplex elektronikát igényel. Három bemenetére
egyesével egy cella kapcsolódik, amelyek között kiegyensúlyozza a terhelést és
kimenetén \textbf{12V} egyenfeszültséget állít elő.

A kaszkád utolsó eleme egy kapcsoló üzemű tápáramkör. A panel bemenete a BMS
által előállított \textbf{12V}, amelyből két buck konverter állít elő
\textbf{6V}-ot és \textbf{5V}-ot. Az \textbf{5V} kimenetből lecsatolva egy LDO
állítja elő a \textbf{3.3V} feszültséget.

A kimeneti feszültségek az alábbi eszközöket hivatottak meghajtani:

\begin{center}
  \begin{tabular}{ | c | c | }
    \hline
    Feszültség & alkatrész 
    \\ \hline
    3.3V & STM32F103 \\
    5V & RaspberryPi \\
    6V & DC motorok
    \\ \hline
  \end{tabular}
\end{center}

\missingfigure{IDE JÖN MAJD LINK A SCHEMATICRA ÉS A LAYOUTRA}


%----------------------------------------------------------------------------
\section{A motor és szenzor vezérlés}
%----------------------------------------------------------------------------

A motorok vezérlése önmagában tekintve komoly feladat amely erősen valósidejű
működést igényel. Erre a célra külön panel készült, amelynek két fő feladata van.
Az egyik, hogy a motorok vezérlését és szabályozását valós időben végezze el, a
másik hogy a fedélzeti (navigációhoz használt) szenzorokhoz és a motorokhoz
egységeb interfészt biztosítson.

\subsection{Hardver}

A feladathoz használt mikrokontroller egy \todo{link a datasheetre}STM32F103
típusú mikrokontroller. A megfelelő mikrovezérlő kiválasztásában több szempont
játszott szerepet.

A kontroller magja ARM Cortex M3 típusú, melynek teljesítménye teljesen elegendő
a feladathoz. Rendelkezik a megfelelő perifériákkal amelyek szükségesek a
szenzorok és motorok kihasználásához Az STMicroelectronics vezérlőivel korábbi
tapasztalat állt már a rendelkezésemre, amely a fejlesztést valamint a megfelelő
ár-érték arányú eszköz kiválasztását felettébb meggyorsította. Végül, de nem
utolsó sorban a fejlesztés ideje alatti gazdasági és piaci viszonyok mellett is
be tudtam szerezni.

A panel a mikrovezérlőn kívül a következő csatlakozási pontokat adja:

\begin{itemize}
\item 2 db csatlakozó a motorokhoz, valamint az ezeken található enkóderekhez
\item 4 db csatlakozó I2C busz interfészhez, amely segítségével szenzorokat illeszthetünk a panelhez
\item 1 db USART interfész debug célra
\item 1 db I2C interfész a Raspberry Pi kapcsolódási pontjaként
\item 1 db stepper motor vezérlő kimenet
\item 1 db tápcsatlakozó bemenet
\item 1 db STLink kompatibilis programozó csatlakozó
\end{itemize}

\missingfigure{Szemléltető ábra a csatlakozókhoz a board felülnézeti képével valamint bekeretezésekkel}

\subsection{Firmware}

A mikrovezérlő firmware megírása során törekedtem a platformfüggetlen, eönnyen
áttekinthető kód készítésére. A teljesítmény maximalizálása és a kódméret
kordában tartása végett nem szerepel a projektben HAL\footnote{HAL: Hardware
Abstraction Layer} kód. Minimális generált kóddal dolgoztam, amely kimerül a
linkerscript, valamint a startup fájlokban, ezeket a gyártó STM32CubeMX nevű
szoftverével hoztam létre. A vezérlő felkonfigurálásához, és perifériáinak
eléréséhez a CMSIS\footnote{CMSIS: Cortex Microcontroller Software Interface
Standard, az ARM által megkövetelt szabványos nevezéktan és támogatás minden
Cortex típusú magot tartalmazó mikrovezérlőre} által szolgáltatott
regiszterdefiníciókat használtam. A firmware megírása során arően támaszkodtam
az STM32F103 \todo{Ez egy hivatkozás az urlre}Reference Manualjára.

A két motor és több szenzor egyidejű kezelése komoly időzítési feladat, és
számtalan versenyhelyzetből adódó hibalehetőséget rejt magában. Ennek a
problémának a csökkentése érdekében egy nyílt forráskódú operációs rendszer
felhasználását láttam indokoltnak. A 32 bites mikrovezérlőkön használatos open
source operációs rendszerek közül a \todo{ez egy hivatkozás a honlapra}FreeRTOS
a legnépszerűbb és legjobban dokumentált, ezért választásom erre az OS-re esett.
A projekt során nagy segítségemre volt \todo{hivatkozás a youtube lejátszási
  listára} Shawn Hymmel oktatóvideó sorozata, amelyben ismereteimet erről a
rendszerről feleleveníthettem. 

A firmware tervezése során több csoportra bontottam a feladatot, amelyeket aztán
task-okba csoportosítva implementálható a firmware.

\missingfigure{Taskokról egy ábra}

A kód pontos funkcionalitása a diplomatervezés kezdetekor még nem volt
kialakítva, így a firmware implementálása már a diplomatervezés feladataiban
szerepel.
