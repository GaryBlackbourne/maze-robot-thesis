%----------------------------------------------------------------------------
\chapter{A linux operációs rendszer}
%----------------------------------------------------------------------------

A Linux operációs rendszer napjainkban alapvető eszköz a legtöbb villamosmérnöki és informatikai területen. Megtalálhatjuk minden nagyobb teljesítményt igénylő informatikai eszközünkben. Számos router és switch Linux rendszert
futtat, de megjelenik kisebb beágyazott eszközökben és szervereken is egyaránt. Napjainkban az asztali felhasználása is egyre inkább teret nyer köszönhetően a személyreszabhatóságának, teljesítményének és ingyenességének.

A Linux eredetileg egy Minix klónként indult amelyet Linus Torvalds Finn származású informatikus kezdett el fejleszteni. Első kiadása 1992-ben jelent meg, azóta rohamos és aktív fejlesztés alatt van. 
Széleskörű népszerűségét számos dolognak köszönheti.


