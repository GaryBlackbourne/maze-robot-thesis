%----------------------------------------------------------------------------
\chapter{Kiinduló projekt}
%----------------------------------------------------------------------------

A diplomatervezés feladat az önálló laboratóriumi munkám folytatása, amelynek
keretein belül egy autonóm robot megtervezése és megépítése volt a cél.
A feladat elvégzése során széleskörű szakmai ismeretekre tettem szert. 

Az autonóm robotok között egy relative könnyen megvalósítható konstrukció a
differenciális robot, így a projekt fő fokuszában egy ilyen konstrukció  mellett
maradtam. Ennek a robot típusnak összesen két motor áll rendelkezésére hogy
helyzetét és orientációját változtassa, ezáltal  könnyebben megvalósítható mint
komplexem meghajtással rendelkező társai, viszont mozgékony és sok lehetőséget
tartogató kialakítás. Klasszikus elrendezése, hogy a két motort a hosszanti
tengellyel merőlegesen kell elhelyezni, ezálal a  a robot sebessége a motorok
szögsebességéből valamint a kerékátmérőből számítható. A robot orientációját a
két motor különböző nagyságban és/vagy különböző sebességgel történő
meghajtásával lehet vezérelni. A kialakítás további előnyeihez tartozik, hogy a
robot vezérléséből csak relative kevés erőforrást kell a mozgásra allokálni,
hiszen pusztán csak a két motorhoz tartozó szabályozó és irányító algoritmusokat
kell futtatni, ugyanakkor a robot sík, vagy közel sík terepen minden különösebb
nehézség nélkül képes navigálni, ezáltal sokféle alkalmazásban ideális választás
lehet.

A projekt egyik fő célja az volt hogy minél átfogóbb ismereteket szerezzek az
egyes témakörökben amiket a robot tervezése során érintek, ezeket a dolgozat
további részeiben rendre dokumentálom.

A robot tervezése során 9 fő feladatot határoztam meg:

\begin{itemize}
\item mechanikai kialakítás megtervezése és kialakítása
\item energiaellátás megtervezése és kialakítása
\item motor és szenzorvezérlő áramkör megtervezése és kialakítása
\item motor és szenzorvezérlő áramkör firmware megtervezése és kialakítása
\item magasabb szintű döntéshozó egység választás
\item beágyazott linux rendszer készítése
\item ROS2 keretrendszer telepítése
\item A saját hardware integrálása a ROS2 keretrendszerbe
\item Demo alkalmazás elkészítése amelyben a robot labirintusban halad.
\end{itemize}

A robot kialakítása során fontos szempont volt a modularitás. Így ha egy
alkatrész meghibásodik, vagy a minősége nem megfelelő, az eszközt könnyebben
lehet frissíteni, fejleszteni. Ezen felül, amennyiben új elvárás merülne fel,
úgy kevés alkatrész cseréjével, vagy újak implementálásával könnyen alkalmassá
tehető a robot, egyéb feladatok ellátására is.

A feladatok könnyebb strukturálása valamint a rendszer pontosabb átláthatósága
érdekében blokkdiagramot is készítettem:

\missingfigure{ide jön majd a blokkdiagram a robotról}

Az önálló laboratórium során több problémába is ütköztem, melyek közül az egyik
legfontosabb a chiphiány, amely jelentősen megnehezítette a  megfelelő alkatrészek
megtalálását és ezzel jelentősen lelassította a tervezési fázist is.

Az önálló laboratórium keretei között végül sikeresen elkészült a tápellátásért
felelős panel, és a motorvezérlést biztosító panel. Döntés született a
felhasználandó motorok, szenzorok, és egyéb alkatrészek ügyében, valamint
elkészült egy kezdetleges firmware amelyel a mikrokontroller perifériáit
tesztelte és egy hozzá tartozó build környezet, amely lehetővé tette számomra,
hogy minden munkát a saját számítógépemen egy linuxos környezetben végezzek.

A projekt ebben az állapotban ért az önálló laboratórium végére, így a hátralévő
feladatok, valamint az újonnan defíniált célok már a diplomaterv feladataiba
tartoznak.


%----------------------------------------------------------------------------
\section{Alkatrészválasztás}
%----------------------------------------------------------------------------

A megfelelő alkatrészek kiválasztása kritikus feladat volt a tervezési szakasz
legelején.

Első lépésben a robot hozzávetőleges tömege és kialakítása alapján választottam
motorokat, amelyek a fő beavatkozó szervek. A beépített enkóderek és a megfelelő
tápfeszültség alapján az n20e motorok mellett döntöttem. Ezek több
konfigurációban is kaphatóak voltak, és inkrementális enkóderrel szerelték fel
őket, így a sebességmérés is lehetővé vált ezeknek a motoroknak a használatakor.
A kereskedő weboldalán a motorokhoz megfelelő felfüggesztést, és tengelyre
szerelhető kereket is találtam, ami a mechanikai szereléseknél nagy segítség
volt. Ezek a motorok kefés DC motorok amelyeket gyakran használnak
hobbielektronikában. A motorok kapocsfeszültsége 6V, valamint a beépített
enkóderek működéséhez 3.3V-tól 5V-íg tetszőlegesen megválasztható feszültségszint
tartozik.

A következő lépésben a távolságérzékeléssel foglaloztam. A távolságérzékelőknek
több fajtája létezik, működési elveikből kifolyólag. A projekthez én az
STMicroelectronics VL53L1 érzékelőjét választottam, amit előzetes kutatásaim
és ajánlások alapján választottam. Ez a szenzor egy LIDAR, lézer alapú
távolságmérő berendezés. A szenzor egyik nagy előnye volt az I2C interfész, amin
a szenzorral kommunikálni lehet. Erős érv volt továbbá a VL53L1-SATEL board, ami
egy kis méretű development board amelyre a szenzor fel van ültetve. A board
egyik felén standard tüskesorhoz tartozó furatok vannak, amelyekre a szenzor 
fontos lábai ki vannak vezetve. A board ennek köszönhetően könnyen felszerelhető,
nem kell hozzá külön panelt tervezni, valamint a tápellátást is a SATEL boardra
lehet bízni. Ezzel a modullal a szenzor könnyen használható 3.3V
tápfeszültségről.

A projekt legelejétől fogva fontos lépés volt, hogy egy beágyazott linux alapú
vezérlő tudjon a magas szintű feladatokkal foglalkozni, hogy a firmware
komplexitása lecsökkenhessen, valamint a rendszer megőrizze modularitását. A
legelterjedtebb megoldás erre a célra a raspberry pi, amely egy ARM alapú
SBC\footnote{Single Board Computer}. Ennek az eszköznek hatalmas népszerűsége
van, valamint rendkívül jól dokumentált, így alkalmasnak éreztem erre a
feladatra.

A fenti specifikációkkal már el tudtam kezdeni a robot topológiájának
megtervezését, valamint a táp, és a vezérlőpanelek megtervezését.

A következő fejezetben bemutatom a robot tápellátásának teljes tervét az
elkészült táp panelt, valamint a vezérlőpanelt.

