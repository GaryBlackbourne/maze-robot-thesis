%----------------------------------------------------------------------------
\chapter{\bevezetes}
%----------------------------------------------------------------------------

Napjaink fejlesztéseinek jelentős része automatizálási feladat, amely gyakran robotok alkalmazását jelenti. Számos
fejlesztés törekszik rá, hogy minél gyorsabb és hatékonyabb robotokat építhessünk, ilyenek a beágyazott processzorokat
érintő fejlesztések, vagy az IoT és 5G kommunikációs hálózatot érintő fejlesztések, amelyek elosztott rendszerek
támogatását hivatottak elősegíteni.

A robotika praktikus megoldásokat kínál számunkra olyan helyzetekben amikor az emberi munkaerő nem alkalmazható,
vagy kiváltásával olcsóbb vagy biztonságosabb munkavégzés válik elérhetővé. Bizonyos feladatok megoldásához
elkerülhetetlen azonban, hogy a robot döntéseket legyen képes hozni a környezet adottságai alapján.
Ez robotok alkalmazásakor komplex feladatnak minősül.

\medskip

Egy robotikai alkalmazásban könnyen előfordulhat olyan szituáció amikor olyan robotra van szükség, amely képes
haladni egy megtervezett pályán és navigálni annak egy kitüntetett pontjára.
Gondolhatunk egy raktár logisztikájára, lakás takarítására, vagy akár egy csatornarendszer karbantartására is.
Egy ilyen robot kialakításánál egyszerre játszik szerepet a méretbeli megkötés, a teljesítményigény valamint a
megbízhatóság.

Egy a fentihez hasonló feladatot modelez a kiírásban szereplő labirintus.

\medskip

A beágyazott rendszerek területén nagyobb teljesítményű processzorok felhasználására, valamint komplex rendszerek
implementálására a legnépszerűbb platform a Linux. Ennek a operációs rendszernek számos előnye van, amely miatt a
fent ismertetett megkötéseknek és igényeknek eleget tesz.

A robot fejlesztése során a magas szintü funkciók kialakítására alkalmas eszköz a Robot Operating System csomag,
amely egy DDS-en alapuló keretrendszer. Moduláris könnyű felépítése és Linux rendszerbe integráltsága ideális
eszközzé teszi, egy fent ismertetett feladathoz.

A Diplomaterv során egy autonóm robot megtervezésének és felprogramozásának a fázisait mutatom be. Elsőként
a projekthez felhasznált alap projektet mutatom be, amelyre a későbiek során építkezni fogok.
Egyesével kitérek a központibb felhasznált eszközök működésére úgy mint a felhasznált hardverek, és szenzorok,
a Linux rendszer és működése, a Yocto project és ROS keretrendszer, majd a projektben történő
felhasználásukat mutatom be.

\todo{ha elkészül összefoglalom hogy melyik fejezetben miről legyen szó}






