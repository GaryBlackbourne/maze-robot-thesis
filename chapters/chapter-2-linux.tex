%----------------------------------------------------------------------------
\chapter{A Linux operációs rendszer}
%----------------------------------------------------------------------------

A Linux operációs rendszer napjainkban alapvető eszköz a legtöbb villamosmérnöki és informatikai területen. Megtalálhatjuk minden nagyobb teljesítményt igénylő informatikai eszközünkben. Számos router és switch Linux rendszert
futtat, de megjelenik kisebb beágyazott eszközökben és szervereken is egyaránt. A mobiltelefonok túlnyomó többsége is Linux alapú operációs rendszert futtat, mert az android operációs rendszer szintén Linux alapú.
Napjainkban az asztali felhasználása is egyre inkább teret nyer köszönhetően a személyre szabhatóságának, teljesítményének és ingyenességének.

A Linux egy Unix-like operációs rendszer. Eredetileg egy Minix klónként indult amelyet Linus Torvalds Finn származású informatikus kezdett el fejleszteni. Első kiadása 1992-ben jelent meg, azóta rohamos és aktív fejlesztés alatt áll.
Széles körű népszerűségét számos dolognak köszönheti.

Az operációs rendszer rendkívül személyre szabható, ami rengeteg különböző igényű alkalmazás számára teszi elérhetővé. Az nyílt forráskódú licensznek\footnote{licensz: GPLv2}
köszönhetően olcsó megoldás, valamint fejlesztését nem egy fejlesztőcsoport, hanem egy kiterjedt közösség végzi, így a hibák keresése sokkal gyorsabban történhet. A fejlesztésben több nagyvállalat, mint például az intel vagy az ARM
is részt vesz biztosítva, hogy  a kernel minél több típusú hardveren futtatható legyen, és minél több eszközhöz legyen driver. A nyílt forráskód ugyanakkor lehetővé teszi saját módosítások, javítások, illetve kiegészítések
eszközölését. Ezen tulajdonság pedig rendkívül hasznos, ugyanis lehetőséget ad a saját hardver kihasználására.

A Linux kernel 2.6.22 verzió óta támogat egy real time működést, amellyel a kernel soft real-time ütemezésre képes. Ez a mód a legtöbb esetben nem javít, sőt ront a rendszer teljesítményén, ellenben bizonyos kritikus beágyazott
feladatok nem megvalósíthatóak real-time működés nélkül.

\section{Az operációs rendszer felépítése}

Egy Linux alapú operációs rendszer általában több részegységből áll, ezek egymást elfedő rétegekként, vagy egymásra épülő szintenként képzelhetőek el. A Linux, pontosabban szólva GNU/Linux önmagában csak a kernelt foglalja magában,
egy teljes rendszer működéséhez azonban több egyéb komponensre is szükség van.

\subsection{Kernel}

A kernel feladata, hogy a rendelkezésre álló erőforrásokat elossza a végrehajtandó feladatok között, valamint platformot adjon ezen feladatok közötti kommunikációnak. A kernel alapja egy ütemező, azaz scheduler. Ennek a
programnak feladata, hogy a számítási teljesítményt és a processzor időt elossza a futtatandó taskok között. Minden task saját virtuális processzoron fut, valamint saját stack-el rendelkezik. Ebből kifolyólag egy task szempontjából
nem észrevehető amikor a scheduler kicserél két futó taskot egy processzoron.

A kommunikációban azonban akadhatnak szinkronizációs problémák is, a fent ismertetett működés következtében. Ebből az okból a kernel feladata megfelelő kommunikációs és szinkronizációs eszközöket adni a taskok számára. Ilyen
eszközök a message queue-k, a shared memory, amelyek üzenetküldésre és kommunikációra használható eszközök,. valamint a mutexek és szemaforok, amelyek a versenyhelyzetből adódó hibákat hivatottak kiküszöbölni, valamint  a taskok
számára szinkronizációs lehetőséget biztosítanak.

A Linux kernel eredetileg monolitikus kernel, azaz egy lefordított oszthatatlan program, amelynek módosításához az egész kernel újrafordítására van szükség. Ez a kialakítás nem optimális hiszen minden új eszközcsatlakoztatása esetén
az eszközhöz tartozó drivert bele kell fordítani a kernelbe. A probléma áthidalása érdekében módosításokat végeztek a Linux kernelen, és egy moduláris kernelt hoztak létre. Ennek a megvalósításához dinamikus linkelésre volt szükség.
A dinamikus linkelés azt jelenti, hogy a program futás közben, külső file tartalmát képes saját magához linkelni, minek következtében a kívülről linkelt funkcionalitás az akciót követően már elérhető a kernelből. A különböző
eszközökhöz használt driverek ilyen módon könnyebben linkelhetőek a kernelhez, ezáltal fejlesztésük és felhasználásuk is roppant dinamikussá vált. Hátulütője a folyamatnak hogy a kernel sebezhető ha instabil modulok kerülnek
betöltésre. Egy rosszul működő modul könnyedén az egész rendszert tönkreteheti, ezért új modul betöltésénél fokozott elővigyázatosságra van szükség.

\subsection{Shell}

A legtöbb rendszerben kritikus, hogy legyen valamilyen felhasználói bemenet, amelyet Unix-like rendszerek esetében általában egy úgynevezett shell biztosít. A shell a kernelre épít, és egy szöveges interfészt biztosít a
felhasználónak, amin keresztül parancsokat adhatunk az operációs rendszernek, valamint programokat indíthatunk, vagy navigálhatunk a fájlrendszerben A rendszer természetesen működhet shell nélkül, de ebben az esetben nincs
módunk az eszközünk megbontása nélkül hozzáférni az operációs rendszerhez, ezért ez csak ritkán fordul elő szinte csak beágyazott környezetben.

Többfajta shell implementáció is létezik, ezek közül a teljesség igénye nélkül csak párat említek:

\begin{itemize}
\item sh
\item bash
\item zsh
\item ksh
\item fish
\end{itemize}

Az első és legrégebbi, ezáltal beágyazott rendszerben leggyakrabban előforduló shell az sh, azaz a Bourne Shell. Ez egy kis méretű shell amely eredetileg a Unix 78. verziójának volt alapértelmezett shellje. Ennek a
shellnek a továbbfejlesztett verziója a bash, azaz Bourne Again Shell, amely új funkciókkal, valamint kényelmi újításokkal látta el elődjét.

A lista többi eleme ritkán fordul elő beágyazott környezetben, leggyakrabban desktopon találkozunk velük. 

\subsection{GUI}

A terminál interfészen kívül a Linux operációs rendszerek képesek grafikus felhasználói felületet is biztosítani. A beágyazott környezetekre kevésbé jellemző, azonban ha valamilyen képi megjelenítést szeretnénk elérni,
gyakran egyszerűbb megoldás egy létező grafikus környezet használata, mint az alapjaiból megírni a kimenetet. Ilyen környezetből két fontos környezetet különíthetünk el. Az egyik az X11 a másik a Wayland.

Az X11 egy régi kiforrott rendszer, amely elfedi a ki és bemeneteket (úgy mint, monitor, egér, billentyűzet). Szerver kliens kapcsolat alapú amely annak a megközelítésnek a maradványa, hogy egy központi számítógéphez több
kisebb gép csatlakozik, a megjelenítést a végpontok végzik, míg a számításigényes feladat a központi rendszeren fut. Ez a megközelítés napjainkban kevésbé praktikus, lévén a legtöbb grafikus alkalmazást ugyanaz a gép jeleníti meg,
mint amelyik a számításokat végzi, ezért felesleges többletműveleteket kell végezni a protocol betartása érdekében.

Fejlesztése lassuló tendenciát mutat, helyét egyre inkább átadja egy új protokollnak a Waylandnek.

\medskip

A wayland egy modernebb, korszerűbb protokol és architektúra, amely a grafikus alkalmazások megjelenítésére lett kifejlesztve. Koncepcióját tekintve egyszerűbb és kisebb az erőforrásigénye. Fontos komponense egy Úgynevezett
wayland compositor, amelyet minden grafikus környezet implementálhat. A wayland biztosít egy nyelvet amin keresztül ezzel a compositor-ral kommunikálhatnak alkalmazásaink.

Az első verzió kiadására 2008-ban került sor. Sajnos a technológia kiforratlansága, valamint az X11-re építő könyvtárak mennyisége miatt az átállás nehézkes.

\section{A boot folyamat}

A könnyebb mikrokontrollereken használt operációs rendszerekkel ellentétben egy Linux alapú rendszer boot folyamata komplexebb, és több időt is vesz igénybe. A boot folyamatot fázisokra bonthatjuk, amelyek időben egymást követik,
és mindegyik fázis az időben utána következőt készíti elő, egészen a teljes rendszer felállásáig

\subsection{BIOS}

A bekapcsolást követően az alaplapba égetett firmware azaz a BIOS\footnote{BIOS: Basic Input Output System} indul el. Ennek a rövid programnak a feladata a POST azaz Power On Self Test öntesztelő eljárás végrehajtása,
majd a megfelelő bootloader betöltése az MBR szerint. Az MBR azaz Master Boot Record az első szektorban található kis terület amely tartalmazza a bootloader információit, valamint a partíciós táblát. Ezek alapján a BIOS elindítja a
megfelelő bootloadert.

\subsection{Bootloader}

A bootloader egy kicsi és kompakt program ami az operációs rendszer komplex indítását hivatott elősegíteni. Képes akár több kernel imaget is kezelni, és fő feladata a kernel image betöltése a memóriába. PC-s környezetben
leggyakrabban a GRUB\footnote{GRUB: Grand Unified BootLoader} nevű bootloaderrel találkozhatunk. A neve kicsit megtévesztő lehet, valójában a GRUB2 programra használt név lévén az első verziót szinte teljesen leváltotta.

Beágyazott környezetben gyakran az U-Boot\footnote{U-Boot: Universal Boot} szolgál erre a célra. Ez a bootloader egy teljes bootloader, amely a teljes bootfolyamatért képes felelni, ellentétben a GRUB-bal, amely csak úgynevezett
'second stage' bootloader. További lényeges különbség, hogy míg a GRUB x86 architektúrára lett tervezve, addig az U-Boot ARM architektúrákra, és ez utóbbi architektúra erősen dominálja a piacot. 

\subsection{Kernel}

A kernel az operációs rendszer magja. Betöltés után egy tömörített speciális fájlrendszert, az initramfs-t betölti a memóriába ami biztosítja a kernel számára a megfelelő eszközöket a rendszer elindításához. A következő lépésben a
kernel további hardver inicializációkat hajt végre majd felcsatolja a root fájlrendszert. utolsó lépésben elindítja az init programot, amely a rendszer használatához szükséges serviceket indítja el.

\subsection{Init}

Az init program feladata a rendszer számára fontos process-ek elindítása. Ez a process egyedüliként a 0 process id-val rendelkezik, és gyökérpontja a processtreenek. Működése scriptekkel konfigurálható, kilenc szintet tartalmaz,
amelyekhez külön indítási script tartozik.

\begin{enumerate}
  \item Leállítás
  \item Egy felhasználós mód
  \item Több felhasználós mód hálózat nélkül
  \item Több felhasználós mód
  \item nem használt
  \item Több felhasználós mód grafikus környezettel
  \item újraindítás
  \item nem használt
  \item nem használt
  \item nem használt
\end{enumerate}

Az init processnek két híresebb implementációja van, a SystemV és a systemd. A systemd újabb, de kompatibilitás miatt a régi futási szintekhez aliasokat tartalmaz. További különbség, hogy a systemd konfigurációkat tárol, amelyek a
shellt hivatottak tehermentesíteni, és a függőségek analizálásával párhuzamosítás segítségével gyorsítják a boot folyamatot.

\section{A Linux fordítása}

A Linux operációs rendszer egyik legnagyobb előnye, hogy az egész rendszer forráskódjához szabad hozzáférésünk van, módosítható és fordítható. Fordítása azonban összetett és bonyolult feladat, amelynek automatizálására
több keretrendszer is épült. Ezek a keretrendszerek általában megkönnyítik a fordítás folyamatát, konfigurációs interfészt biztosítanak a kernelhez, illetve adnak olyan eszközöket, amelyek segítségével a végső image könnyedén
elkészíthető.

Két fontos rendszerről érdemes megemlékezni, a Buildroot-ról, valamint a Yocto Projectről. A két build rendszer között az egyik legnagyobb különbség a fordításra használt fő eszközben rejlik. A buildroot egy script és Makefile
gyűjtemény amely segít a forráskód lefordításában, valamint a root fájrendszer generálásában. Mindkét rendszer keresztfordítást használ, hogy a célhardveren futó binárisokat állítson elő.

A projekt során nagyobb komplexitása ellenére a yocto projectet választottam, amely szabadabb fejlesztést és személyre szabhatóságot tesz lehetővé. További nagy előnye hogy képes felhasználni az előzőm fordítás kimenetét, így hiba,
vagy kisebb javítás után nem kell a teljes folyamatot előröl kezdeni. A következő fejezetben részletesen foglalkozom vele.



